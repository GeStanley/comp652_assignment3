\documentclass{report}
% Change "article" to "report" to get rid of page number on title page
\usepackage{amsmath,amsfonts,amsthm,amssymb}
\usepackage{setspace}
\usepackage{Tabbing}
\usepackage{fancyhdr}
\usepackage{lastpage}
\usepackage{extramarks}
\usepackage{chngpage}
\usepackage{soul,color}
\usepackage{listings}
\usepackage{enumerate}
\usepackage{graphicx,float,wrapfig}
\usepackage{pifont}
\usepackage{graphicx}
\usepackage[english]{babel}
\usepackage{tikz}
\usepackage[]{algorithm2e}
% In case you need to adjust margins:
\topmargin=-0.45in      %
\evensidemargin=0in     %
\oddsidemargin=0in      %
\textwidth=6.5in        %
\textheight=9.0in       %
\headsep=0.25in         %

\title{Assignment 1 - Comp 652 - Machine Learning}

% Homework Specific Information
\newcommand{\hmwkTitle}{Assignment 3}                     % Adjust this
\newcommand{\hmwkDueDate}{Tuesday, March 31 2015}                           % Adjust this
\newcommand{\hmwkClass}{COMP 652}


\newcommand{\hmwkClassInstructor}{Dr. Doina Precup}
\newcommand{\hmwkAuthorName}{Geoffrey Stanley}
\newcommand{\hmwkAuthorNumber}{260645907}
\newcommand{\Pp}{\mathbb{P}}
\newcommand{\Ev}{\mathbb{E}}
\newcommand{\cov}{\text{Cov}}
\newcommand{\Z}{\mathbb{Z}}
\newcommand{\R}{\mathbb{R}}
\newcommand{\dd}{\, \mathrm{d}}

% Setup the header and footer
\pagestyle{fancy}                                                       %
\lhead{\hmwkAuthorName}                              %
\chead{}
\rhead{\hmwkClass: \hmwkTitle}                                          %

\lfoot{}
\cfoot{}                                                                %
\rfoot{Page\ \thepage\ of\ \pageref{LastPage}}                          %
\renewcommand\headrulewidth{0.4pt}                                      %
\renewcommand\footrulewidth{0.4pt}                                      %

% This is used to trace down (pin point) problems
% in latexing a document:
%\tracingall
\definecolor{mygreen}{rgb}{0,0.6,0}
\lstset{commentstyle=\color{mygreen}, frame=single,  language=R, showspaces=false, showstringspaces=false}

%%%%%%%%%%%%%%%%%%%%%%%%%%%%%%%%%%%%%%%%%%%%%%%%%%%%%%%%%%%%%
% Make title
\title{\vspace{2in}\textmd{\textbf{\hmwkClass:\ \hmwkTitle}}\\
\normalsize\vspace{0.1in}\small{Due\ on\ \hmwkDueDate}\\
\vspace{0.1in}\large{\textit{Presented to \hmwkClassInstructor}}\vspace{3in}}
\date{}
\author{\textbf{\hmwkAuthorName}\\
    \textbf{Student ID: \hmwkAuthorNumber}}
%%%%%%%%%%%%%%%%%%%%%%%%%%%%%%%%%%%%%%%%%%%%%%%%%%%%%%%%%%%%%

\begin{document}
\maketitle
\section*{Question 1}
\subsection*{A)}
In the 4-neighbor spin glass model the maximal cliques were the edges between
each pixel. In an 8-neighbor spin glass model the maximal cliques become
clusters of 4 pixels.
\subsection*{B)}

\subsection*{C)}

\section*{Question 2}
As dimension are reduced from 250 the reconstruction error is initially quite
small but becomes exponentially more important as dimensions approach 0. The
shoulder of the reconstruction error line is at a dimension of approximately X.
From this we can conclude that X features are necessary to model the relationship
between the features and the target in this dataset.
\section*{Question 3}
\subsection*{A)}
As with standard Hidden Markov Models, Coupled Hidden Markov models will
have three categories of parameters. These are the initial probabilities, the
transition probabilites and the emission probabilities. Given the system
depicted in Figure 1 of assignment 3:\\

Initial Probabilities:
\begin{equation}
  P(s0)
\end{equation}
\begin{equation}
  P(u0)
\end{equation}\\

Transition Probabilities:
\begin{equation}
  P(s_i | s_{i-1}, u_{i-1})
\end{equation}
\begin{equation}
  P(u_i | u_{i-1}, s_{i-1})
\end{equation}\\

Emission Probabilities:
\begin{equation}
  P(y_i | s_i)
\end{equation}
\begin{equation}
  P(z_i | u_i)
\end{equation}

\subsection*{B)}
The solution is to use a foward algorithm.

\subsection*{C)}

\subsection*{D)}
The algorithm will remain largely the same except in the definition of the model
parameters. More specifically, the initial and transition probabilities will need
to be altered. K initial probabilities will be required and transition probabilities
would be defined as follows in the current model context:

\begin{equation}
  P(s_i | s_{i-k}, u_{i-k})
\end{equation}
\begin{equation}
  P(u_i | u_{i-k}, s_{i-k})
\end{equation}
\subsection*{E)}

\end{document}
